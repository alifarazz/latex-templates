\documentclass[a4paper,11pt]{article}
\usepackage{verbatim} % fancy comments
% refs
\usepackage{hyperref}
\hypersetup{pdftex,colorlinks=true,allcolors=red}
\usepackage{hypcap}

% define the title
\author{A.~Faraz}
\title{How to become \TeX{}perts}
\begin{document}
% generates the title
\maketitle
\tableofcontents

\maketitle
\pagebreak

\section{Paragraph}
%\section*{Basics}

\paragraph{}
This is a new paragraph. \\Here's a new-line! \newline Here's another one!

\paragraph{}
This is another paragraph.

\paragraph{Paragraph Title}
This is a paragraph with a title.

\pagebreak
\section{Commands}
Layout: \textbackslash command[optional parameter]\{\emph{parameter}\} \\ 
You can \textsl{lean} on me! \\
Please, start a new-line here!\newline Thank you!

\paragraph*{Special Chars}\mbox{}\\
\# \$ \% \^{} \& \_ \{ \} \~{}
\textbackslash

\pagebreak
\section{Comments}
This is % a rather stupid
% Better: an instructive <----
an example: ``This will appear% but after the comment won't'
!''	\newline
This is another
%% needs package \usepackage{verbatim}
\begin{comment}
	rather stupid,
	but helpful
\end{comment}
example for embedding comments in your document.\footnote[1]{But this one is a footnote!}

\pagebreak
\section{Environment \& Flush}
%\paragraph{Environment}
\flushleft
\begin{enumerate}
	\item You can nest the list
	environments to your taste:
	\begin{itemize}
		\item But it might start to
		look silly.
		\item[-] With a dash.

	\end{itemize}
	\item Therefore remember:
	\begin{description}
		\item[Stupid] things will not
		become smart because they are
		in a list.
		\item[Smart] things, though,
		can be presented beautifully
		in a list.
	\end{description}
\end{enumerate}

\paragraph{Flush}
\begin{itemize}
	\item \begin{flushleft}
		This text is\\ left-aligned.
		\LaTeX{} is not trying to make
		each line the same length.
	\end{flushleft}
	\item \begin{flushright}
		This text is\\ right-aligned.
		\LaTeX{} is not trying to make
		each line the same length.
	\end{flushright}
	\item \begin{center}
		At the center \\ of the earth.
	\end{center}
\end{itemize}
\pagebreak
\section{Tabular}
\paragraph{General}
\begin{center}
\begin{tabular}{|r|l|}
	\hline
	7C0 & hexadecimal \\ \cline{1-1}
	3700 & octal \\ \cline{2-2}
	11111000000 & binary \\
	\hline \hline
	1984 & decimal \\
	\hline
\end{tabular}
\end{center}
\mbox{}\\
\paragraph{Custom Columns}
\begin{center}
\begin{tabular}{@{} r @{~~} l @{}}
	\hline
	7C0 & hexadecimal \\
	3700 & octal \\
	11111000000 & binary \\
	\hline
\end{tabular}
\end{center}

\mbox{}\\
\paragraph{Multi Column}
\begin{center}
\begin{tabular}{|c|c|}
	\hline
	\multicolumn{2}{|c|}{States} \\
	\hline
	False & True \\
	\hline
\end{tabular}
\end{center}

\mbox{}\\
\paragraph{Align Around Decimal Point}
\begin{center}
	\begin{tabular}{c r @{.} l}
		Pi expression
		&
		\multicolumn{2}{c}{Value} \\
		\hline
		$\pi$
		& 3&1416 \\
		$\pi^{\pi}$
		& 36&46
		\\
		$(\pi^{\pi})^{\pi}$ & 80662&7 \\
	\end{tabular}
\end{center}


\end{document}
