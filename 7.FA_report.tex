\documentclass[12pt]{report}

\usepackage[obeyspaces]{url}
\usepackage{graphicx}
\usepackage{xepersian}
\usepackage[T1]{fontenc}
\usepackage[utf8]{inputenc}
\usepackage[]{verbatim}



\settextfont[Scale=1]{XB Kayhan}
\setlatintextfont[Scale=1]{Droid Sans}
\setdigitfont[Scale=1]{XB Zar}

% Title Page
\title{فاز اول پروژه:  FreeBSD}
\author{محمد علی فرزدقی ۹۵۳۲۲۸۴}


\begin{document}
\maketitle

%\begin{abstract}
%\end{abstract}

\section{ساختن کرنل}
\paragraph{دانلود}
ابتدا از سایت
\lr{freebsd.org }،
فایل
\lr{ .vmdk }
 که از قبل برای استفاده در
\lr{  VM}
   آماده شده را دانلود می‌کنیم و به عنوان یک
\lr{   Storage Device }
   به
\lr{    VM}
     اضافه می‌کنیم.


\paragraph{راه اندازی}
در ابتدا وقتی به صفحه‌ی بوت رسیدیم، حالت
\lr{ multi-user }
 را انتخاب می‌کنیم.
	\begin{figure}[h!]
		\centering
		\includegraphics[width=.7\linewidth]{coffee.jpg}
		\caption{صفحه‌ی بوت}
		\label{fig:bootscreen}
	\end{figure}\\
بعد از بوت شدن کامل سیستم عامل به عنوان 
\lr{root }
لاگین می‌کنیم و پکج 
\lr{subversion}
 را نصب می‌کنیم.
	\begin{figure}[h!]
		\centering
		\includegraphics[width=.7\linewidth]{coffee.jpg}
		\caption{نصب svn}
		\label{fig:svn_inst}
	\end{figure}
	
	\pagebreak
	
	
\paragraph{دانلود سورس کد ها}
به پوشه‌ی
 \path{/usr/src }
 می‌رویم و با کمک ،
\lr{subversion}
  شروع به دانلود سورس کد
\lr{freebsd}
    می‌کنیم.
	\begin{figure}[h!]
		\centering
		\includegraphics[width=.7\linewidth]{coffee.jpg}
		\caption{دانلود سورس ها}
		\label{fig:svn_checkout}
	\end{figure}
	
\paragraph{کانفیگ کرنل}
در فایل
\path{/usr/src/sys/amd64/conf}
کانفیگ های کرنل برای معماری \lr{amd64} وجود دارد. ما از کانفیگ پیش فرض 
\lr{GENERIC}
 برای بیلد کردن کرنل استفاده می‌کنیم.
\paragraph{بلید کردن کرنل}
با صدا زدن \lr{make} دادن پارامتر \lr{buildkernel} به آن، کرنل را با کانفیگ \lr{GENERIC} بیلد می‌کنیم.

	\begin{figure}[h!]
		\centering
		\includegraphics[width=.7\linewidth]{coffee.jpg}
		\caption{دستور برای کامپایل}
		\label{fig:compile_kern}
	\end{figure}

\pagebreak


\paragraph{نصب کردن کرنل}
نصب کردن هم همانند بیلد کردن کرنل است ولی پارامتر فرق می‌کند. به جای \lr{buildkernel}، از \lr{installkernel} استفاده می‌کنیم.

	\begin{figure}[h!]
		\centering
		\includegraphics[width=.7\linewidth]{coffee.jpg}
		\caption{بعد از نصب موفقیت آمیز}
		\label{fig:kernel_installed}
	\end{figure}
	
	حال دفعه بعدی که سیستم را روشن می‌کنیم، کرنل کامپایل شده خودمان شروع به بوت شدن می‌کند.
%%%%%%%%%%%%%%%%%%%%%%%%%%%%%%%%%%%5
\pagebreak
\section{تغییر دادن \lr{Priority}}
\paragraph{پیش نیاز ها}
در کرنل
 \lr{freebsd}
  المانی که پردازش را انجام می‌دهد یک 
 \lr{thread}
   است که در غالب یک
 \lr{lwp}،
     قسمتی از یک
 \lr{process}
      را تشکیل می‌دهد.
 \paragraph{دستورالعمل تغییر دادن \lr{Priority}}
 \begin{itemize}
 	\item ابتدا با استفاده از 
	 	\lr{pid} 
	 	مربوط \lr{process}، خود \lr{process} را پیدا می‌کنیم. فانکشن
	 	 pfind 
	 	 مناسب این کار است.
	 	 \footnote[1]{\path{src/sys/kern/kern_proc.c}~~~~~~~~~~~~\path{src/sys/sys/proc.h}}
	 	 
    \item با استفاده از 
	 	\lr{process} 
	 	به دست آمده در مرحله قبل،
	   \lr{thread} 
	  مربوطه را پیدا می‌کنیم. فانکشن 
	  find\underline~thread
	   مناسب این کار است.
	 	 \footnote[2]{\path{src/sys/kern/kern_thread.c}~~~~~~~~\path{src/sys/sys/proc.h}}
	 	 
	\item با استفاده از 
	 	\lr{thread} 
	 	به دست آمده در مرحله قبل، می‌توانیم با استفاده از فانکشن
	 	 setpriority\underline~sys
	، \lr{priority} مربوط به 
	\lr{thread} را تغییر بدیهم.
	 	 \footnote[3]{\path{src/sys/kern/kern_resource.c}~~~~\path{src/sys/sys/sysproto.h}}
 \end{itemize}
 
 \pagebreak
 \section{انواع \lr{Prioriy}}
 هر چه \lr{priority} کمتر باشد، اولویت آن \lr{thread} بیشتر است. در کل ۲ کلاس از \lr{thread} ها وجود دارند، \lr{thread} های \lr{kernel} و \lr{user}.\\
 در جدول، کل دسته بندی های \lr{prioriy} مربوط به \lr{thread} آمده است. در زمان اجرا کلاس های دسته‌ی بالاتر به کلاس های دسته‌‌ی پایین‌تر ترجیح داده می‌شوند.
 \lr{\begin{center}
		 \begin{tabular}{@{~~~~~~~} c @{~~~~~~~} c @{~~~~~~~} l @{~~~~~~}}
			 \hline
			 type Thread & Class & Prioriy~~~~\\
			 (interrupt) kernel buttom-half & ITHD & 47 - 0~~~~~\\
			 user real-time & REALTIME & 79 - 48~~~\\
			 kernel top-half & KERN & 119 - 80~~~\\
			 user time-sharing & TIMESHARE & 223 - 120\\
			 user idle & IDLE & 255 - 224\\
			 \hline
		\end{tabular}
	\end{center}}

\section{\lr{Prioriy Inheritance}}
درکل وقتی یکی از دستور های
 \lr{vfork}، 
 \lr{fork }
 یا 
 \lr{rfork }
 در کرنل اجرا می‌شوند، \lr{process} ای که حاصل می‌شود \lr{priority} یکسانی با \lr{process} به وجود آورنده اش دارد.

\end{document}          
