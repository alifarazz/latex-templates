\documentclass[a4paper,11pt]{article}
\usepackage{amsmath}
\usepackage{array} % format tabular by cm
% define the title
\usepackage{hyperref}
\hypersetup{pdftex,colorlinks=true,allcolors=red}
\usepackage{hypcap}
\author{A.~Faraz}
\title{\AmS{} Package}
\begin{document}
% generates the title
\maketitle
% insert the table of contents
\tableofcontents
\pagebreak
\section{Basics}
\paragraph{Simple}
	Add $a$ squared and $b$ squared
	to get $c$ squared. Or, using
	a more mathematical approach:
	$a^2 + b^2 = c^2$
	
%tags + label + eqref
\paragraph{Single-line}
	Add $a$ squared and $b$ squared
	to get $c$ squared. Or, using
	a more mathematical approach
	\begin{equation}
	a^2 + b^2 = c^2
	\end{equation}
	Einstein says
	\begin{equation}
	E = mc^2 \label{clever}
	\end{equation}
	He didn’t say
	\begin{equation}
	1 + 1 = 3 \tag{dumb}
	\end{equation}
	This is a reference to
	\eqref{clever}.

\paragraph{Without Numbers}
	Add $a$ squared and $b$ squared
	to get $c$ squared. Or, using
	a more mathematical approach
	\begin{equation*}
	a^2 + b^2 = c^2
	\end{equation*}
	or you can type less for the
	same effect:
	\[ a^2 + b^2 = c^2 \]
	
\pagebreak
\section{Formula}
\paragraph{Sums \& Lim} 
\mbox{}\\
This is text style:
$\lim_{n \to \infty}
\sum_{k=1}^n \frac{1}{k^2}
= \frac{\pi^2}{6}$.
And this is display style:
\begin{equation}
\lim_{n \rightarrow \infty}
\sum_{k=1}^n \frac{1}{k^2}
= \frac{\pi^2}{6}
\end{equation}

\paragraph{Matrix}
\[
\begin{matrix}
a & b & c \\
d & e & f \\
g & h & i
\end{matrix}
\]
\begin{equation*}
M = \begin{bmatrix}
	\frac{5}{6} & \frac{1}{6} & 0           \\[0.3em]
	\frac{5}{6} & 0           & \frac{1}{6} \\[0.3em]
	0           & \frac{5}{6} & \frac{1}{6}
\end{bmatrix}
\end{equation*}
\begin{equation*}
	\begin{matrix}
		1 & 2 \\
		3 & 4
	\end{matrix} \qquad
	\begin{bmatrix}
		p_{11} & p_{12} &
		\ldots
		& p_{1n} \\
		p_{21} & p_{22} &
		\ldots
		& p_{2n} \\
		\vdots & \vdots &
		\ddots
		& \vdots \\
		p_{m1} & p_{m2} &
		\ldots
		& p_{mn}
	\end{bmatrix}
\end{equation*}

\paragraph{Integral}
\[ F(s) = \int_{0}^{\infty}{f(t)e^{-st}}\,dt \]
\[ S(x) = \iint f(x).g(y).h(z)\,dy\,dz\]
\paragraph{Formula + Matrix}
~
% h -> here
% b -> buttom of page
% t -> top of page
% p -> on extra page
% ! -> override(will force specified location)
\begin{table}[h!]
	\centering
	\begin{tabular}{| m{6.1cm} | m{6.1cm} |}
	\hline
	\multicolumn{2}{|c|}{laplace}
	\\
	\hline
	\multicolumn{1}{|c|}{$t \rightarrow s$} & \multicolumn{1}{c|}{$s \rightarrow t$} \\
	\hline
	$F(s) = \int_{0}^{\infty}{f(t).e^{-st}}\,dt$ &
	$f(t) = \newline \frac{1}{2 \pi i}\lim_{T \to \infty} \int_{\gamma - iT}^{\gamma + iT} F(s).e^{st}\, ds$ \\
	\hline  
	\end{tabular}
\end{table}

\end{document}

